\documentclass[12pt,a4j]{jarticle}


\title {第8回CIR-KIT定期ミーティング\\ロボットアームの関節機構}
\author {下松八重 宏太}
\date {\today}

\begin{document}
\maketitle

\section{はじめに}
ARCのロボットアーム制作にあたって、まずはロボットアームの機構について考
える必要がある。昨年のトマトロボット競技会においても様々な形のロボットアー
ムが参加していた。そこで、今回はロボットアームの関節機構についてまとめよ
うと思う。

\section{ロボットアームの関節と自由度}
ロボットアームを開発する場合、まずはその自由度について考える必要がある。
自由度とは、平面及び空間内においてアームがどの方向に動くことができるのか
を表す尺度である。
この時、方向とはアームの傾きと位置を意味する。
通常、ロボットアームの自由度は各関節の数と一致する。また、今回のように空
間内で作業をするロボットアームは6自由度あれば十分であるとされる。ただし、
今回はロボットアームの台座となるものが自立移動ロボットであるため、アーム自体の自由度は4自由度あればよいと考えられる。(表\ref{tab:arm})
 
\begin{table}[htb]
\begin{center}
  \caption{関節数と自由度}
  \begin{tabular}{|c|c|c|} \hline
  アーム&関節数 &自由度 \\ \hline \hline
  空間作業用ロボットアーム&6&x,y,z軸+x,y,z回転=6
	  自由度\\ \hline
  今回開発するロボットアーム& 4&x,y,z軸+1軸回転=4自由度\\ \hline
  人間の腕&3& 肩3 + 肘1 + 手首3 = 7自由度\\ \hline
  \end{tabular}
\label{tab:arm}
\end{center}
\end{table}

\end{document}


\newpage
\section{ロボットアームの関節機構}
\documentclass[12pt,a4j]{jarticle}



\title {第8回CIR-KIT定期ミーティング\\ロボットアームの関節機構}
\author {下松八重 宏太}
\date {\today}

\begin{document}
\maketitle

\section{はじめに}
ARCのロボットアーム制作にあたって、まずはロボットアームの機構について考える必要がある。昨年のトマトロボット競技会においても様々な形のロボットアームが参加していた。そこで、今回はロボットアームの関節機構についてまとめようと思う。

\section{ロボットアームの関節と自由度}
ロボットアームを開発する場合、まずはその自由度について考える必要がある。
自由度とは、平面及び空間内においてアームがどの方向に動くことができるのかを表す尺度である。
この時、方向とはアームの傾きと位置を意味する。
通常、ロボットアームの自由度は各関節の数と一致する。また、今回のように空間内で作業をするロボットアームは6自由度あれば十分であるとされる。ただし、
今回はロボットアームの台座となるものが自律移動ロボットであるため、アーム自体の自由度は4自由度あればよいと考えられる。(表\ref{tab:arm})
 
\begin{table}[htb]
\begin{center}
  \caption{関節数と自由度}
  \begin{tabular}{|c|c|c|} \hline
  アーム&関節数 &自由度 \\ \hline \hline
  空間作業用ロボットアーム&6&x,y,z軸+x,y,z回転=6自由度\\ \hline
  今回開発するロボットアーム& 4&x,y,z軸+1軸回転=4自由度\\ \hline
  人間の腕&3& 肩3 + 肘1 + 手首3 = 7自由度\\ \hline
  \end{tabular}
\label{tab:arm}
\end{center}
\end{table}


\newpage


\section{ロボットアームの関節機構}
ロボットアームの関節機構は大きく分けて回転関節と並進関節の2種類がある。並進関節とは主に油圧式シリンダーやネジ式直動機などを指す。その名の通り関節が直進して前後することでアームを操作するタイプの関節である。
回転関節とは我々がロボットアームと聞いて最も先に思いつくであろう関節のタイプである。回転関節は曲げ関節とねじり関節に分けられる。リンク方向に対して、回転軸が垂直についている回転関節を曲げ関節と呼ぶ。回転軸が平行についている回転関節をねじり関節と呼ぶ。ただし、リンク機構の形によって、曲げ関節もねじり関節も同じ働きが出来る為、特別に両者を区別したりしない。
\begin{table}[htb]
 \begin{center}
  \caption{回転関節と並進関節の長所と短所}
  \begin{tabular}{|c|c|c|} \hline
   各項目&回転関節&並進関節\\ \hline \hline
   安定性&低い &高い \\ \hline
   制御性&難解 &易い \\ \hline
   作業空間&比較的自由 &限定される \\ \hline
   駆動装置&小型のもので良い &大型の駆動軸が必要 \\ \hline
  \end{tabular}
 \end{center}
\end{table}
\end{document}
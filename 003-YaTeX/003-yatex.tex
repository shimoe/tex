% --------------------------------------
%   文書クラス
% --------------------------------------
\documentclass[12pt,a4j]{jarticle}

% --------------------------------------
%   パッケージ
% --------------------------------------
\usepackage{amsmath,amssymb}
\usepackage{bm}
\usepackage{graphicx}
\usepackage{ascmac}

% --------------------------------------
%   余白の調整
% --------------------------------------

% --------------------------------------
%   マクロの定義
% --------------------------------------

% --------------------------------------
%   \maketitle
% --------------------------------------
\title{第三回ミーディング TeXの導入}
\author{山崎達也}
\date{\today}

% --------------------------------------
%   本文
% --------------------------------------
\begin{document}
\maketitle
\section{はじめに}
今週の進捗は全く無いが,\LaTeX の導入方法を簡単に纏めておいてくれとの声
があったので今回の報告はこれについてとする.文法その他についてはネット
に詳しい資料があるので,内容はインストール手順と注意事項に留める.\LaTeX
がどのようなものかという事についても同様である.
 
\section{インストール}
\TeX ,\LaTeX 共にパッケージがリポジトリにあるのでaptを使用して簡単にイン
ストールが出来る.必要なコマンドは以下の通りである.

{
\small
\begin{verbatim}
$ sudo apt-get -y install texlive
$ sudo apt-get -y install texlive-lang-cjk
$ sudo apt-get -y install xdvik-ja
$ sudo apt-get -y install dvipsk-ja
$ sudo apt-get -y install gv
$ sudo apt-get -y install texlive-fonts-recommended texlive-fonts-extra
\end{verbatim}
}

\section{YaTeXの導入}
以上の様にインストールしただけでも十分使えるが,このままでは確認の度に端
末からコンパイルしなければならない.
幸いCIR-KITの開発メンバーはEmacsを主エディタとしており,Emacs用\LaTeX 入力
支援環境としてYaTeXというものがあるのでそれの使用をお勧めする.

{
\small
\begin{verbatim}
$ sudo apt-get -y install yatex
\end{verbatim}
}

これも同じくaptを使用して以上の様にインストール,Emacs側に以下のコードを
追加すればファイル名末尾が.texのファイルの編集で自動的にYaTeXモードが起
動する.

{
\small
\begin{verbatim}
(setq auto-mode-alist
  (cons (cons "\\.tex$" 'yatex-mode) auto-mode-alist))
(autoload 'yatex-mode "yatex" "Yet Another LaTeX mode" t)
\end{verbatim}
}

コンパイルするためにはEmacs上でC-c C-t j,プレビューはC-c C-t pで利用で
きる.その他にも補完などがあるが,これを覚えるだけでも作業効率が飛躍的に高
まる.作成からプレビューまでが簡単になるだけでPDFへの変換はYaTeX導入後もコマンドから行う必要が
ある.作成中のtexファイルをC-c C-t jでコンパイルした後,出力されたdviファイル
を変換する.コマンドは以下の通り.

{
\small
\begin{verbatim}
$ dvipdfmx example.div
\end{verbatim}
}

尚,注意事項としてEmacsのYaTeX用のLispをinit.elとは別のファイルとする場
合,ディレクトリ・ファイル名に単にYaTeXという語を使うとYaTeXが起動しなく
なる事に注意すること(小文字のyatexでも同様).\\
 文字化け等の問題があった場合は聞いて下さい.



\end{document}